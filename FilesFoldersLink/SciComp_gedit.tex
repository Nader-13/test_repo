% Type of document
\documentclass[letterpaper,12pt]{article}

% Packages
\usepackage[dvips]{graphicx}
\usepackage{amsmath,amssymb,color,fancybox,setspace}
\usepackage[numbers]{natbib}

% Page format
\setlength{\oddsidemargin}{0.00in}  % Margin on the odd numbered side
\setlength{\evensidemargin}{0.00in} % Margin on the even numbered side
\setlength{\topmargin}{-0.50in}     % Margin from the top 
\setlength{\textwidth}{6.50in}      % Width of the text in a page
\setlength{\textheight}{9.00in}     % Height of the text in a page
\setlength{\parindent}{0.00in}      % New paragraph indentation
\setlength{\parskip}{0.20in}        % Spacing between two paragraphs

% Page style
\pagestyle{plain}

% Document begins
\begin{document}

Per Wikipedia, scientific computing is a multi-disciplinary field that 
uses advanced computing capabilities to understand and solve complex 
problems. It is the application of computer simulation to solve problems
in various arts, science and engineering disciplines -- to gain an 
understanding mainly through the analysis of mathematical models 
implemented on computers.

The field includes algorithms and modeling/simulation software developed
to solve the problems, computer and information science that develops and
optimizes the system (hardware, software, networking, and data management,
etc.) needed to solve the problems, and the computing infrastructure that
supports both the science and engineering problem solving and the 
developmental computer and information science.

% Document ends
\end{document}
